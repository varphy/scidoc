\documentclass[10pt,a4paper]{article}
\usepackage[utf8]{inputenc}
\usepackage[ngerman]{babel}
\usepackage{amsmath}
\usepackage{amsfonts}
\usepackage{amssymb}
\usepackage{mathtools}
\usepackage{microtype}
\usepackage{authblk}
\usepackage{titlesec}

% section title formating
\renewcommand{\thesection}{\Roman{section}}
\titleformat{\section}[block]{\normalsize\bfseries}{\thesection.}{1em}{\MakeUppercase}
\titlespacing*{\section}{0pt}{11pt plus 2pt minus 2pt}{11pt plus 2pt minus 2pt}

\renewcommand{\thesubsection}{\arabic{section}.\Alph{subsection}}
\titleformat{\subsection}{\normalsize\bfseries}{\thesubsection}{1em}{}
\titlespacing*{\subsection}{0pt}{11pt plus 2pt minus 2pt}{11pt plus 2pt minus 2pt}

\renewcommand{\thesubsubsection}{\thesubsection.\arabic{subsubsection}}
\titleformat{\subsubsection}{\normalsize\itshape}{\thesubsubsection}{1em}{}
\titlespacing*{\subsubsection}{0pt}{11pt plus 2pt minus 2pt}{11pt plus 2pt minus 2pt}
   
% custom macros
\newcommand{\mydef}[1]{\textls{#1}}

% %%%%%%%
% title %
% %%%%%%%

\title{Kurzeinführung in lineare Zusammenhänge}
\author{Hendrik Meier und Jean-Philippe Dugard}

\affil{Vernon, Frankreich}
\renewcommand\Affilfont{\itshape}

\date{August 2022}


% %%%%%%%%%%
% document %
% %%%%%%%%%%

\begin{document}
\maketitle

In diesem Aufsatz führen wir den Zusammenhangs\-begriff auf (reellen) Vektorbündeln ein.
Im Folgenden sei dazu $M$ eine glatte Mannigfaltigkeit der Dimension~$\mathrm{dim}\ M = n$ und $B$ ein glattes Vektorbündel über~$M$ vom Rang~$m$ mit Projektion
\begin{align*}
\pi:B\rightarrow M
\ .
\end{align*}
Für einen Punkt~$x\in M$ ist die Faser~$\pi^{-1}(x)$ also ein Vektorraum der Dimension~$m$, den wir mit~$B_x$ bezeichnen.
Die Mannigfaltigkeit~$M$ nennen wir den \mydef{Grundraum} des Vektorbündels.
Als \mydef{Vektorfeld}~$X$ bezeichnen wir in diesem Aufsatz einen glatten Schnitt im Vektorbündel~$B$, also eine glatte Abbildung~$X:M\rightarrow B$ mit~$\pi\circ X = \mathrm{id}_M$.\\

Betrachten wir das Vektorfeld~$X$ in einem Punkt~$x\in M$, wo es den Wert~$X|_x$ annimmt.
\emph{Wir interessieren für eine quantitative Angabe darüber, wie sich~$X$ von $x$ aus in Richtung eines gegebenen Tangential\-vektors $\xi\in \mathrm{T}_xM$ ändert.} 
Die gesuchte Angabe sollte sinnvollerweise ein Vektor in~$B_x$ sein, der linear von $\xi$ abhängt.
Wir wählen die übliche Bezeichnung
\begin{align*}
\left.\nabla_\xi X\right|_x
\end{align*}
für diesen Vektor.

Wählen wir als Ausgangspunkt die Ableitung des Vektorfeldes~$X$ im Punkte~$x$,
\begin{align*}
dX|_x : \mathrm{T}_xM \rightarrow \mathrm{T}_{X|_x}B
\ ,
\end{align*}
müssen wir feststellen, dass~$dX|_x$ alleine unser eingangs formuliertes Interesse nicht befriedigt, da wir ohne Weiteres nicht wissen, wie wir einem Tangential\-vektor~$\beta\in \mathrm{T}_{X|_x} B$ einen Vektor in~$B_x$ zuordnen sollen.
Unser erstes Ziel ist die Konstruktion von Strukturen, die eine solche Zuordnung eindeutig ermöglichen.

\newpage

\section{Zusammenhänge auf Vektorbündeln}

Ziel dieses Abschnitts ist die Definition, Konstruktion und Darstellung von Zusammenhängen auf einem Vektorbündel.
Hinsichtlich der Bezeichnungen und Symbole übernimmt folgender Text jene, die in der Einleitung eingeführt worden sind.

\subsection{Der vertikale Raum}
Betrachten wir zunächst den mit Hilfe der Bündelprojektion~$\pi$ definierten $m$-dimensionalen Untervektorraum
\begin{align*}
\mathrm{V}_{X|_x}B = \mathrm{Kern}\ d\pi|_{X|_x} \subset \mathrm{T}_{X|_x} B
\ .
\end{align*}
Dieser sogenannte \mydef{vertikale} Raum ist kanonisch isomorph zu~$B_x$.

Zur Begründung dieser Aussage definieren wir für einen gegebenen Vektor~$v_x\in B_x$ den Weg
\begin{align*}
\gamma_v: \mathbb{R} &\rightarrow B
\ ,\nonumber\\
t &\mapsto X|_x + tv_x
\ ,
\end{align*}
der vollständig in $B_x\subset B$ enthalten ist.
Es ist~$\gamma_v(0) = X|_x$.
Die Abbildung
\begin{align*}
\kappa_{X|_x}: B_x&\rightarrow\mathrm{T}_{X|_x}B
\ ,\nonumber\\
v_x &\mapsto \left.\frac{d\gamma_v}{dt}\right|_{t=0}
\end{align*}
ist offenbar ein injektiver Vektorraum\-homomorphismus.\footnote{Dies zeigt sich z.B. unmittelbar in lokalen Karte einer lokalen Trivialisierung.} 
Sein Bild ist ein Unterraum von $\mathrm{V}_{X|_x}B$, dem Kern von~$d\pi|_{X|_x}$, wie die Kettenregel
\begin{align*}
d\pi|_{X|_x} . \left.\frac{d\gamma_v}{dt}\right|_{t=0}
= 
\left.\frac{d}{dt}
\left(
\pi\circ\gamma_v
\right)
\right|_{t=0} = 0
\end{align*}
zeigt.
(Man beachte, dass $t\mapsto\pi\circ\gamma_v=x$ ein konstanter Weg ist.) 
Aus Dimensions\-gründen muss $\mathrm{Bild}\ \kappa_{X|_x}$ mit $\mathrm{V}_{X|_x}B$ sogar übereinstimmen.
Wir bestätigen die behauptete kanonische Isomorphie
\begin{align*}
B_x \cong \mathrm{V}_{X|_x}B
\end{align*}
also mit Hilfe der Abbildung~$\kappa_{X|_x}$.

\subsection{Horizontale Räume und Zusammenhänge}
\label{ssec:horizontal}

Zur Konstruktion von~$\left.\nabla_\xi X\right|_x$ fehlte an dieser Stelle nur noch eine Projektion
\begin{align*}
p_V|_{X|_x}: \mathrm{T}_{X|_x} B\rightarrow\mathrm{V}_{X|_x}B
\ ,
\end{align*}
die die Ableitung~$dX|_x.\xi$ in Richtung~$\xi$ zunächst auf~$\mathrm{V}_{X|_x}B$ projizierte, von wo aus die Umkehrung der Isomorphie~$\kappa_{X|_x}$ zurück auf $B_x$ führte.
Mit~$p_V|_{X|_x}$ hätten wir somit alle Zutaten für eine in~$\xi$ lineare Abbildung mit den gewünschten Eigenschaften in unserer Hand.

Eine Projektion~$p_V|_{X|_x}$ definiert eine direkte Summen\-zerlegung
\begin{align*}
\mathrm{T}_{X|_x}B = \mathrm{V}_{X|_x}B \oplus \mathrm{H}_{X|_x}B
\end{align*}
mit dem \mydef{horizontalen} Raum $\mathrm{H}_{X|_x}B=\mathrm{Kern}\ p_V|_{X|_x}$.
Umgekehrt liefert die Auszeichnung eines horizontalen Raums~$\mathrm{H}_{X|_x}B$ als Komplement zum kanonischen Unterraum~$\mathrm{V}_{X|_x}B$ eindeutig die Projektion~$p_V|_{X|_x}$.

Die Auszeichung\footnote{\label{note:existence}Jedes ($C^\infty$-)Vektorbündel lässt eine solche Auszeichnung zu. Jedes ($C^\infty$-)Vektorbündel kann sogar mit einem \emph{linearen} Zusammenhang, siehe \ref{ssec:linear}, versehen werden. Ein solcher kann z.B. aus trivialen Zusammenhängen in lokalen Trivialisierungen gewonnen, die mit Hilfe einer Zerlegung der Eins zu einem glatten globalen linearen Zusammenhang zusammengesetzt werden.} eines zum Bündel der vertikalen Räume~$\mathrm{V}_{X|_x}B$ komplementären \mydef{Horizontal\-bündels}
\begin{align*}
\mathrm{H}B = \coprod_{X|_x} \mathrm{H}_{X|_x}B
\end{align*}
stattet jeden Tangential\-raum~$\mathrm{T}_{X|_x}B$ des Vektorbündels~$B$ mit einer Projektion~$p_V|_{X|_x}$ aus.
Damit können wir nun den gesuchten \mydef{Zusammenhang} in der Form
\begin{align}
\label{eq:connection}
\left.\nabla_\xi X\right|_x
= (\kappa_{X|_x}^{-1}\,|\,\mathrm{V}_{X|_x}B)\,.\,
p_V|_{X|_x}\,.\,
dX|_x
\,.\,\xi
\end{align}
als Verkettung von Abbildungen
\begin{align*}
\mathrm{T}_xM 
\rightarrow \mathrm{T}_{X|_x}B
\xrightarrow{p_V|_{X|_x}} \mathrm{V}_{X|_x}B
\rightarrow B_x
\end{align*}
angeben.
Ein Zusammenhang ist also durch ein ausgezeichnetes Horizontalbündel~$\mathrm{H}B$ eindeutig definiert.

\subsection{Konstruktion von Zusammenhängen}
\label{ssec:constr}

Sei~$\nabla$ ein Zusammenhang auf dem Vektorbündel~$B$ und wiederum~$x\mapsto X|_x$ ein Vektorfeld.
Der durch den Zusammenhang definierte horizontale Raum~$\mathrm{H}_{X|_x}B$ wird durch die Ableitung~$d\pi|_{X|_x}$ surjektiv und damit aus Dimensions\-gründen isomorph auf den Tangential\-raum~$\mathrm{T}_xM $ abgebildet.
Die Abbildung
\begin{align*}
p_H|_{X|_x}: \mathrm{T}_{X|_x}B &\rightarrow \mathrm{H}_{X|_x}B
\ ,\nonumber\\
\beta &\mapsto  (d\pi|_{X|_x}\,|\, \mathrm{H}_{X|_x}B)^{-1}\,.\,d\pi|_{X|_x}\,.\beta
\end{align*}
ist die zu $p_V|_{X|_x}$ komplementäre Projektion, so dass 
\begin{align*}
p_V|_{X|_x} + p_H|_{X|_x} = \mathrm{id}|_{\mathrm{T}_{X|_x}B}
\end{align*}
eine Darstellung der Identität auf~$\mathrm{T}_{X|_x}B$ ist.

Sei nun~$\tilde{\nabla}$ ein weiterer Zusammenhang auf~$B$, der auf~$\mathrm{T}_{X|_x}B$ die Projektionen~$\tilde{p}_V|_{X|_x}$ und~$\tilde{p}_H|_{X|_x}$ liefert.
Aus der Definition des Zusammenhangs in~\ref{ssec:horizontal} erhalten wir für die Differenz beider Zusammenhänge den Ausdruck
\begin{align*}
\tilde{\nabla}_\xi X|_x - \nabla_\xi X|_x 
&= (\kappa_{X|_x}^{-1}\,|\,\mathrm{V}_{X|_x}B)\,.\,
\left(\tilde{p}_V|_{X|_x} - p_V|_{X|_x}\right)\,.\,
dX|_x
\,.\,\xi
\ .
\end{align*}
Durch Einsetzen der Identität auf~$\mathrm{T}_{X|_x}B$ erhalten wir zunächst
\begin{align*}
dX|_x
&= 
\left(
p_V|_{X|_x} + p_H|_{X|_x}
\right)\,.\,
dX|_x
\nonumber\\
&= 
\left[
p_V|_{X|_x} +
(d\pi|_{X|_x}\,|\, \mathrm{H}_{X|_x}B)^{-1}\,.\,d\pi|_{X|_x}
\right]\,.\,
dX|_x
\nonumber\\
&= p_V|_{X|_x}\,.\,dX|_x
+ (d\pi|_{X|_x}\,|\, \mathrm{H}_{X|_x}B)^{-1}
\,.\,\overset{= \mathrm{id}_{\mathrm{T}_xM}}{\overbrace{d(\pi\circ X)|_x}}
\nonumber\\
&= p_V|_{X|_x}\,.\,dX|_x
+ (d\pi|_{X|_x}\,|\, \mathrm{H}_{X|_x}B)^{-1}
\ .
\end{align*}
Mit der Komposition von Projektionen~$\tilde{p}_V|_{X|_x}\circ p_V|_{X|_x} = p_V|_{X|_x}$  folgt
\begin{align*}
\tilde{\nabla}_\xi X|_x - \nabla_\xi X|_x 
&= \alpha(X|_x)\,.\,\xi
\ ,
\end{align*}
wobei die lineare Abbildung~$\alpha(X|_x):\mathrm{T}_{x}M\rightarrow B_x$ durch
\begin{align*}
\alpha(X|_x) =
(\kappa_{X|_x}^{-1}\,|\,\mathrm{V}_{X|_x}B)\,.\,
\left(\tilde{p}_V|_{X|_x} - p_V|_{X|_x}\right)\,.\,
(d\pi|_{X|_x}\,|\, \mathrm{H}_{X|_x}B)^{-1}
\end{align*}
gegeben ist.

Ist auf einem Vektorbündel~$B$ also ein Zusammenhang~$\nabla$ erklärt, können alle anderen Zusammenhänge auf~$B$ in der Form
\begin{align}
\label{eq:connection_alpha}
\nabla^\alpha = \nabla + \alpha
\end{align}
angegeben werden.
Dabei ist die Abbildung
\begin{align*}
\alpha: B\times_M \mathrm{T}M &\rightarrow E
\ ,\nonumber\\
(v_x, \xi_x) &\mapsto \alpha(v_x)\,.\,\xi_x
\end{align*}
in der zweiten Komponente linear.
Das Vektorbündel~$B\times_M \mathrm{T}M$ bezeichnet hier das Faserprodukt\footnote{Auf das Faserprodukt kommen wir in~\ref{ssec:pullback} zurück.} der Vektorbündel~$\mathrm{T}M$ und~$B$ über~$M$ bezüglich deren Projektionen.

\subsection{Zusammenhänge in Koordinatendarstellung}
\label{ssec:coords}

Betrachten wir das Vektorbündel~$B$ für einen vorgegebenen Punkt in~$M$ in einer lokalen Trivialisierung~$U\times V$, wobei~$U\subset M$ eine offene Umgebung dieses Punkts und~$V$ ein~$m$-dimensionaler Vektorraum ist.
Ein Vektorfeld auf $U\times V$
\begin{align*}
X: U &\rightarrow U \times V
\ ,\nonumber\\
x &\mapsto (x, X|_x)
\end{align*}
fassen wir entsprechend als Graphen einer Abbildung~$U\rightarrow V$ auf.
Im Folgenden verwenden wir das Symbol~$X$ für diese Abbildung.

Das triviale Vektorbündel~$U\times V$ trägt den trivialen Zusammenhang
\begin{align*}
\nabla^0 X|_x = dX|_x
\ ,
\end{align*}
wobei~$dX|_x:\mathrm{T}_xU\rightarrow \mathrm{T}_xV \cong V$ hier das gewöhnliche totale Differenzial zwischen Vektorräumen bezeichnet.
Ein allgemeiner Zusammenhang lässt sich nun nach dem Ergebnis von~\ref{ssec:constr} also in der Form
\begin{align*}
\nabla^\alpha X|_x = dX|_x + \alpha(x, X|_x)
\end{align*}
schreiben, wobei $\alpha$ eine glatte Abbildung~$U \times V\rightarrow \mathrm{Hom}(\mathrm{T}_xU, V)$ ist.\\

Um den Zusammenhang~$\nabla^\alpha$ in Koordinaten anzugeben, parametrisieren wir~$U$ --- nach eventueller Verkleinerung in eine kleinere Umgebung von~$x$ --- in einer geeigneten Karte.
Nehmen wir also direkt an, dass $U$ eine offene Menge von~$\mathbb{R}^n$ und~$x\in U\subset\mathbb{R}^n$ ist.
Für den Vektorraum~$V$ können wir gleichermaßen nach Wahl einer geeigneten Basis annehmen, dass $V=\mathbb{R}^m$ ist.


In expliziter Koordinatendarstellung, $x=(x^1, ..., x^n)$, $\xi=(\xi^1, ..., \xi^n)$ und $X|_x= (X^1(x), ..., X^m(x))$, erhalten wir
\begin{align}
\label{eq:connection_coords}
\left(\nabla^\alpha_\xi X|_x\right)^i
= \sum_{j=1}^n 
\left(
\left.
\frac{\partial X^i}{\partial x^j}
\right|_x
+ \alpha^i_j(x, X|_x)
\right)\,\xi^j
\end{align}
für $i=1, ...,m$.
Die Koeffizienten\-funktionen~$\alpha^i_j$ sind dabei die \mydef{Christoffel-Symbole} des Zusammenhangs~$\nabla^\alpha$ in der betrachteten lokalen Trivialisierung und Karte.

\newpage

\section{Lineare Zusammenhänge}

In diesem Abschnitt führen wir lineare Zusammenhänge ein.
Lineare Zusammenhänge existieren auf allen glatten Vektorbündeln, vgl. die Fußnote~\ref{note:existence} auf Seite~\pageref{note:existence}.

\subsection{Lineare Zusammenhänge}
\label{ssec:linear}

Ein Zusammenhang~$\nabla$ ist genau dann \mydef{linear}, wenn in jeder lokalen Trivialisierung~$U\times V$ des Vektorbündels~$B$ die durch
\begin{align*}
\nabla X|_x = dX|_x + \alpha(x, X|_x)
\end{align*}
definierte Abbildung~$\alpha(x, X|_x)$ linear in ihrem zweiten Argument~$X|_x$ ist.
In lokalen Koordinaten einer solchen lokalen Trivialisierung erhalten wir also die Darstellung
\begin{align}
\label{eq:lin_connection_coords}
\left(\nabla_\xi X|_x\right)^i
= \sum_{j=1}^n 
\left(
\left.
\frac{\partial X^i}{\partial x^j}
\right|_x
+ 
\sum_{k=1}^m\Gamma^i_{kj}(x)\,X^k(x)
\right)\,\xi^j
\end{align}
für $i=1, ...,m$.
Dabei werden die \mydef{Christoffel-Symbole}~$\Gamma^i_{kj}$ definiert durch
\begin{align*}
\alpha^i_j(x, v) = \sum_{k=1}^m\Gamma^i_{kj}(x)\,v^k
\end{align*}
für $v=(v^1,...,v^m)\in \mathbb{R}^m$ ($=V$ in der betrachteten Karte).\\

\subsection{Rechenregeln}

Ein linearer Zusammenhang~$\nabla$ erfüllt, wie man nach in einer lokalen Trivialisierung von~$B$ oder direkt in der in~\ref{ssec:linear} angegebenen Koordinaten\-darstellung leicht zeigt, die folgenden charakteristischen Eigenschaften oder Rechenregeln: Sind~$X,Y:M\rightarrow B$ Vektorfelder mit Werten in~$B$, ferner $\Xi,\Pi:M\rightarrow \mathrm{T}M$ Vektorfelder im Tangential\-bündel, $f,g:M\rightarrow \mathbb{R}$ Funktionen auf~$M$ und~$a,b\in\mathbb{R}$, so gilt
\begin{align*}
\nabla_\Xi (aX + bY)
&= a\nabla_\Xi X + b\nabla_\Xi Y
\quad&(\text{\mydef{Linearität}})
\ ,\\
\nabla_{f\Xi+g\Pi}X
&= f\nabla_\Xi X + g\nabla_\Pi X
\quad&(\text{\mydef{$C^\infty$-Linearität}})
\ ,\\
\nabla_\Xi (fX)
&= f\,\nabla_\Xi X + (df\,.\,\Xi)\,X 
\quad&(\text{\mydef{Produktregel}})
\ .
\end{align*}
Umgekehrt definiert eine Abbildung
\begin{align*}
\Gamma(\mathrm{T}M)\times\Gamma(B)\rightarrow\Gamma(B)
\end{align*}
(wobei $\Gamma(-)$ den Raum der Vektorfelder bezeichnet) mit obigen Eigenschaften  eindeutig einen linearen Zusammenhang.

\newpage

\section{Pull-Back-Zusammenhänge}

Eine glatte Abbildung einer Mannigfaltigkeit in den Grundraum eines Vektorbündels~$B$ definiert durch Zurücknehmen (engl. Pull-Back) ein Vektorbündel vom gleichen Range über der Urbild\-mannigfaltigkeit.
Wie sich ein Zusammenhang auf~$B$ auf dieses zurückgenommene Vektorbündel übertragen lässt, ist der Gegenstand dieses Abschnitts.
Von besonderem Interesse ist der Spezialfall von Vektorfeldern entlang einer Kurve im Grundraum eines Vektorbündels.

\subsection{Zurücknehmen eines Zusammenhangs}
\label{ssec:pullback}

Sei~$I$ eine Mannigfaltigkeit der Dimension~$p$ und~$\gamma: I\rightarrow M$ eine glatte Abbildung in die Mannigfaltigkeit~$M$, über der ein Vektorbündel~$B$ vom Rang~$m$ mit Projektion~$\pi$ erklärt ist.
Auf den interessanten Spezialfall, in dem~$I$ ein offenes Intervall in~$\mathbb{R}$ und $\gamma$ somit ein Weg in~$M$ ist, gehen wir in Abschnitt~\ref{ssec:covariant_derivative} ein.
Punkte in~$I$ nennen wir in dieser Voraussicht bereits~$t$ und Tangential\-vektoren an diesen Punkten $\tau=\tau_t\in\mathrm{T}_tI$.

Durch Zurücknehmen mit der Abbildung~$\gamma$ erhält man das Vektorbündel~$\gamma^*B$ (von gleichem Range~$m$) über dem Grundraum~$I$.
Es handelt sich dabei um das Faserprodukt
\begin{align*}
\gamma^*B = I \times_M B = 
\left\{
(t, v_x) \in I\times B\ |\ \gamma(t)=\pi(v_x)=x
\right\}
\end{align*}
bezüglich~$\gamma$ und~$\pi$.
Die Projektion $\gamma^*B\rightarrow I$ des Pull-Back-Vektorbündels ist gegeben durch
\begin{align*}
p_I: (t, v_x)\mapsto t
\ ,
\end{align*}
also einfach die Projektion auf die erste Komponente.

Die Projektion auf die zweite Komponente nennen wir~$p_B$, also $p_B(t,v_x)=v_x\in B_x$ für $x=\gamma(t)$.
Für festes~$t$ ist 
\begin{align*}
p_B(t,-): (\gamma^*B)_t\rightarrow B_{\gamma(t)}
\end{align*}
offenbar ein Vektorraum\-isomorphismus.\\

Ist nun auf~$B$ ein Zusammenhang~$\nabla$ erklärt, können wir diesen in natürlicher Weise zu einem Zusammenhang~$\gamma^*\nabla$ auf dem Bündel~$\gamma^*B$ zurücknehmen.
Den Zusammenhang~$\gamma^*\nabla$ nennen wir auch den \mydef{Pull-Back-Zusammenhang} bezüglich~$\gamma$.

Sei dazu $Y:I\rightarrow\gamma^*B$ ein Vektorfeld auf~$I$, d.h. ein Schnitt in~$\gamma^*B$, und sei $\tau\in\mathrm{T}_tI$ ein an den Punkt~$t\in I$ angehefteter Tangential\-vektor.
Die gewöhnliche Ableitung~$dY|_t$ des Vektorfelds bildet~$\tau$ in den Tangential\-raum $\mathrm{T}_{Y|_t}(\gamma^*B)$ ab, von wo die Ableitung der Faserprodukt\-projektion
\begin{align*}
dp_B|_{Y|_t}:
\mathrm{T}_{Y|_t}(\gamma^*B)
\rightarrow 
\mathrm{T}_{p_B(Y|_t)}B
\end{align*}
in einen Tangential\-raum an $B$ führt.
Der Zusammenhang~$\nabla$ auf~$B$ liefert nun eine Projektion auf dessen Vertikal\-raum und damit auf~$B_{\gamma(t)}=p_B(t,(\gamma^*B)_t)$.
Insgesamt erhalten wir also, vgl. Gl.~(\ref{eq:connection}) in \ref{ssec:horizontal}, in natürlicher Weise die Formel
\begin{align}
\label{eq:pullback}
(\gamma^*\nabla)_\tau Y|_t
= 
[p_B(t,-)]^{-1}\,.\,
(\kappa_{p_B(Y|_t)}^{-1}\,|\,\mathrm{V}_{p_B(Y|_t)}B)\,.\,
p_V|_{p_B(Y|_t)}\,.\,
dp_B|_{Y|_t}\,.\,
dY|_t
\,.\,\tau
\end{align}
für den Zusammenhang~$\gamma^*\nabla$.\\

Sei nun, wie in~\ref{ssec:constr}, mit~$\tilde{\nabla}$ ein weiterer Zusammenhang auf~$B$ gegeben.
Für dessen Pull-Back~$\gamma^*\tilde{\nabla}$ können wir eine zu Gl.~(\ref{eq:pullback}) analoge Formel angeben oder aber versuchen, diesen durch~$\gamma^*\nabla$ auszudrücken: Eine zu jener in~\ref{ssec:constr} analoge Rechnung
\begin{align*}
&dp_B|_{Y|_t}\,.\,dY|_t
\nonumber\\
&= \left(
p_V|_{p_B(Y|_t)} + p_H|_{p_B(Y|_t)}
\right)\,.\,
dp_B|_{Y|_t}\,.\,dY|_t
\nonumber\\
&= 
\left[
p_V|_{p_B(Y|_t)} +
(d\pi|_{p_B(Y|_t)}\,|\, \mathrm{H}_{p_B(Y|_t)}B)^{-1}\,.\,d\pi|_{p_B(Y|_t)}
\right]\,.\,
dp_B|_{Y|_t}\,.\,dY|_t
\nonumber\\
&= 
p_V|_{p_B(Y|_t}\,.\,dp_B|_{Y|_t}\,.\,dY|_t
+ (d\pi|_{p_B(Y|_t}\,|\, \mathrm{H}_{p_B(Y|_t}B)^{-1}
\,.\,d(\pi\circ p_B\circ Y)|_t
\nonumber\\
&= 
p_V|_{p_B(Y|_t}\,.\,dp_B|_{Y|_t}\,.\,dY|_t
+ (d\pi|_{p_B(Y|_t}\,|\, \mathrm{H}_{p_B(Y|_t}B)^{-1}
\,.\,d\gamma|_t
\end{align*}
(da $\pi\circ p_B\circ Y=\gamma$) liefert
\begin{align*}
(\gamma^*\tilde{\nabla})_\tau Y|_t
- (\gamma^*\nabla)_\tau Y|_t
&=
\alpha_\gamma(p_B(Y|_t))\,.\,d\gamma|_t\,.\,\tau
\ .
\end{align*}
Darin ist
\begin{align*}
\alpha_\gamma(p_B(Y|_t) = [p_B(t,-)]^{-1}\,.\,\alpha(p_B(Y|_t))
\end{align*}
eine Abbildung $\mathrm{T}_{\gamma(t)}M \rightarrow (\gamma^*B)_t$, wobei $\alpha$ mit der gleichnamigen Abbildung aus Gl.~(\ref{eq:connection_alpha}) übereinstimmt.\\

Zum Abschluss geben wir den zurückgenommenen Zusammenhang~$\gamma^*\nabla$ in lokalen Koordinaten um einen gegebenen Punkt in~$I$ an.
Sei dazu $U\times V$ wie in Abschnitt~\ref{ssec:coords} eine lokale Trivialisierung und~$J\subset I$ eine offene Umgebung des betrachteten Punkts derart, dass das Bild $\gamma(J)$ vollständig in~$U\subset M$ enthalten ist.
Ein Vektorfeld im (trivialen) Bündel $\gamma^*(U\times V)=I\times_U V=I\times V$ hat wie zuvor die Form eines Graphen,
\begin{align*}
Y: I&\rightarrow I\times V
\ ,\nonumber\\
t&\mapsto (t, Y|_t)
\ ,
\end{align*}
zu einer Abbildung~$I\rightarrow V$, für die wir im Folgenden das Symbol~$Y$ verwenden.
Ferner bemerken wir, dass die Abbildung~$p_B(t,-)$ die Identität auf~$V$ ist, im Urbild aufgefasst als an~$t$ angeheftet und im Bild an~$\gamma(t)\in U$.

Bezüglich des durch
\begin{align*}
\nabla^0Y|_t = dY|_t: \mathrm{T}_tI \rightarrow V
\end{align*}
gegebenen trivialen Zusammenhangs auf $\gamma^*(U\times V)$ können wir also einen vorgegebenen Pull-Back-Zusammenhang~$\gamma^*\nabla^\alpha$ in der Form
\begin{align*}
(\gamma^*\nabla^\alpha)|_t = 
dY|_t 
+ \alpha(\gamma(t), Y|_t)\,.\,d\gamma|_t
\end{align*}
angeben, wobei~$\alpha$ dieselbe glatte Abbildung~$U \times V\rightarrow \mathrm{Hom}(\mathrm{T}_{\gamma(t)}M, V)$ wie in~\ref{ssec:coords} bezeichnet.

In expliziter Koordinatendarstellung, $t=(t^1, ..., t^p)$, $\gamma(t)=(\gamma^1(t), ..., \gamma^n(t))$, $\tau=(\tau^1, ..., \tau^p)$ und $Y|_t= (Y^1(t), ..., Y^m(t))$, erhalten wir
\begin{align}
\label{eq:pullback_coords}
\left(\gamma^*\nabla^\alpha_\tau Y|_t\right)^i
= \sum_{j=1}^p 
\left(
\left.
\frac{\partial Y^i}{\partial t^j}
\right|_x
+ \sum_{k=1}^n 
\alpha^i_k(\gamma(t), Y|_t)\,\frac{\partial\gamma^k}{\partial t^j}
\right)\,\tau^j
\end{align}
für $i=1, ...,m$.
Die Koeffizienten\-funktionen~$\alpha^i_j$ sind dabei wie zuvor die Christoffel-Symbole des Zusammenhangs~$\nabla^\alpha$ in der betrachteten Karte der lokalen Trivialisierung~$U\times V$.

Im Falle eines linearen Zusammenhangs gilt entsprechend
\begin{align}
\label{eq:pullback_coords_lin}
\left(\gamma^*\nabla^\alpha_\tau Y|_t\right)^i
= \sum_{j=1}^p 
\left(
\left.
\frac{\partial Y^i}{\partial t^j}
\right|_t
+ \sum_{k=1}^n\sum_{l=1}^m 
\Gamma^i_{lk}(\gamma(t))\ Y^l(t)\, 
\left.\frac{\partial\gamma^k}{\partial t^j}\right|_t
\right)\,\tau^j
\end{align}
mit den Christoffel-Symbolen~$\Gamma^i_{lk}$ aus \ref{ssec:linear}.

\subsection{Ableitung entlang einer Kurve}
\label{ssec:covariant_derivative}

Sei $I\subset\mathbb{R}$ ein offenes Intervall und $\gamma:I\rightarrow M$ ein Weg in der Mannigfaltigkeit~$M$, die der Grundraum des Vektorbündels $\pi:B\rightarrow M$ sei.

Ein \mydef{Vektorfeld längs~$\gamma$} ist eine Abbildung~$Y:I\rightarrow B$ mit der Eigenschaft $\pi\circ Y=\gamma$ und entspricht damit einem Schnitt im Pull-Back-Bündel $\gamma^*B=I\times_M B$.
Diese Definition schließt den interessanten Spezialfall ein, in dem~$Y$ die Komposition~$X\circ\gamma$ des Weges~$\gamma$ mit einem Vektorfeld~$X$ im Bündel~$B$ ist.\\

Sei $\nabla$ ein Zusammenhang auf~$B$.
Der Pull-Back-Zusammenhang~$\gamma^*\nabla$ definiert eine (Zeit-)Ableitung nach~$t$, die wir die \mydef{kovariante Ableitung} entlang des Weges~$\gamma$ nennen und die durch
\begin{align*}
\left.\frac{\nabla}{dt} Y\right|_t
= \frac{\gamma^*\nabla_\tau Y|_t}{\tau}
\end{align*}
für $\tau\in\mathrm{T}_tI=\mathbb{R}$ mit $\tau\neq 0$ gegeben ist.
In lokalen Koordinaten erhalten wir aus den Gln.~(\ref{eq:pullback_coords}) und~(\ref{eq:pullback_coords_lin}) die Formel
\begin{align*}
\left.\frac{\nabla}{dt} Y^i\right|_t
=  
\left.
\frac{\partial Y^i}{\partial t}
\right|_x
+ \sum_{k=1}^n 
\alpha^i_k(\gamma(t), Y|_t)\,\dot{\gamma}^k
\end{align*}
im allgemeinen Falle sowie im Falle eines linearen Zusammenhangs
\begin{align*}
\left.\frac{\nabla}{dt} Y^i\right|_t
= 
\left.
\frac{\partial Y^i}{\partial t}
\right|_t
+ \sum_{k=1}^n\sum_{l=1}^m 
\Gamma^i_{lk}(\gamma(t))\ Y^l(t)\ 
\dot{\gamma}^k
\ ,
\end{align*}
wobei $\dot{\gamma}^k = \partial\gamma^k/\partial t$.


\end{document}