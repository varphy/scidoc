\documentclass[10pt,a4paper]{article}
\usepackage[utf8]{inputenc}
\usepackage[ngerman]{babel}
\usepackage{amsmath}
\usepackage{amsfonts}
\usepackage{amssymb}
\usepackage{mathtools}
\usepackage{microtype}
\usepackage{authblk}
\usepackage{titlesec}

% section title formating
\renewcommand{\thesection}{\Roman{section}}
\titleformat{\section}[block]{\normalsize\bfseries}{\thesection.}{1em}{\MakeUppercase}
\titlespacing*{\section}{0pt}{11pt plus 2pt minus 2pt}{11pt plus 2pt minus 2pt}

\renewcommand{\thesubsection}{\arabic{section}.\Alph{subsection}}
\titleformat{\subsection}{\normalsize\bfseries}{\thesubsection}{1em}{}
\titlespacing*{\subsection}{0pt}{11pt plus 2pt minus 2pt}{11pt plus 2pt minus 2pt}

\renewcommand{\thesubsubsection}{\thesubsection.\arabic{subsubsection}}
\titleformat{\subsubsection}{\normalsize\itshape}{\thesubsubsection}{1em}{}
\titlespacing*{\subsubsection}{0pt}{11pt plus 2pt minus 2pt}{11pt plus 2pt minus 2pt}
   
% custom macros
\newcommand{\mydef}[1]{\textls{#1}}
\newcommand{\nul}{\textrm{---}}
\newcommand{\trans}{^\mathsf{T}}

% %%%%%%%
% title %
% %%%%%%%

\title{Herleitung der Transportrate}
\author{Hendrik Meier}

\affil{Immenstaad am Bodensee}
\renewcommand\Affilfont{\itshape}

\date{Januar 2023}


% %%%%%%%%%%
% document %
% %%%%%%%%%%

\begin{document}
\maketitle

In diesem Aufsatz wird die sogenannte \mydef{Transportrate} berechnet, die die Rotation von Koordinaten in den NED-Bezugssystem längs einer Trajektorie beschreibt.
Diese Rechnung wird mit elementaren Methoden durchgeführt.

\section{Koordinatensysteme}

\subsection{Erdfeste Koordinaten}

Im erdfesten \mydef{geozentrischen Koordinatensystem (ECEF)} zeichnen wir als Ursprung den Mittelpunkt eines um die Polarachse rotationssymmetrischen Referenzellipsoiden (z.B. WGS84) aus und wählen ein kartesisches Achsensystem, in dem die dritte Achse zum Nordpol führt und die beiden ersten Achsen in der Äquatorialebene liegen.
Im Folgenden identifizieren wir unseren Anschauungsraum mit dem ECEF-Koordinatenraum und entsprechend einen Punkt $r$ mit einem ECEF-Koordinatentupel~$(x,y,z)\in\mathbb{R}^3$ in einer vorgegebenen Längeneinheit.

Neben dem erdfesten linearen Koordinatensystem betrachten wir ein ebenso erdfestes System von sphärischen Koordinaten, den sogenannten \mydef{geogaphischen Koordinaten}, hier in der Konvention $\sigma = (\phi, \lambda, h)$ mit der geographischen Breite~$\phi$, der geographischen Länge~$\lambda$ und der negativen (!) Höhe~$h$ bezüglich der Ellipsoid\-oberfläche.

Die Kartenwechselabbildung zwischen ECEF-Koordinaten $r=(x,y,z)$ und geographischen Koordinaten $\sigma = (\phi, \lambda, h)$ ist durch
\begin{align}
\label{eq:transition_map_geo2ecef}
	r &= \Phi(\sigma)
	=
	\begin{pmatrix}
		\left[
			N(\phi) - h
		\right]
		\cos\phi\cos\lambda
		\\
		\left[
			N(\phi) - h
		\right]
		\cos\phi\sin\lambda
		\\
		\left[
			(1 - \varepsilon^2)N(\phi) - h
		\right]
		\sin\phi
	\end{pmatrix}	
\end{align}
gegeben.
Darin ist
\begin{align}
\label{eq:N}
	N(\phi)
	&=
	\frac{a}{\sqrt{1 - \varepsilon^2\sin^2\phi}}
\end{align}
der längs der Normalen zum Ellipsoiden gemessene Abstand zur Polarachse und $\varepsilon=\sqrt{1-b^2/a^2}$ die Exzentrizität.
Mit $a$ ist hier die Länge der äquatorialen (großen) Halbachse, mit $b$ die Länge der polaren (kleinen) Halbachse bezeichnet.

Zur Umrechnung von (Tangential-)Vektoren in den jeweiligen Koordinatensystemen benötigen wir ferner die Ableitung der Kartenwechselabbildung~$\Phi$. Wir finden
\begin{align}
\label{eq:D_Phi}
	D\Phi(\sigma; \nul)
	&=
	\left[
		M(\phi) - h
	\right]
	\begin{pmatrix}
		-\sin\phi\cos\lambda \\
		-\sin\phi\sin\lambda \\
		\cos\phi
	\end{pmatrix}
	d\phi
	\nonumber\\
	&\quad
	+
	\left[
		N(\phi) - h	
	\right]\cos\phi
	\begin{pmatrix}
		-\sin\lambda \\
		\cos\lambda \\
		0
	\end{pmatrix}
	d\lambda
	+
	\begin{pmatrix}
		-\cos\phi\cos\lambda \\
		-\cos\phi\sin\lambda \\
		-\sin\phi
	\end{pmatrix}
	dh
	\ .
\end{align}
Die Funktion
\begin{align}
\label{eq:M}
	M(\phi) &= \frac{1 - \varepsilon^2}{1 - \varepsilon^2\sin^2\phi} N(\phi)
\end{align}
ist dabei der Krümmungsradius des Meridians.

\subsection{NED-Koordinaten}

Für einen gegebenen Punkt $\sigma = (\phi, \lambda, h)$ ist die Basis $(\partial/\partial\phi,\partial/\partial\lambda,\partial/\partial h)$ offenbar eine Orthogonalbasis bezüglich des (Standard-)Skalarprodukts der ECEF-Koordinaten.
Durch Normierung erhalten wir eine Orthonormalbasis und bezüglich dieser die \mydef{NED($\sigma$)-Koordinaten} (Nord-Ost-Unten) zum Punkt~$\sigma$.

Sei~$u$ ein (Tangential-)Vektor am Punkt~$\Phi(\sigma)$ unseres Anschauungsraums.
Wie zuvor identifizieren wir~$u$ mit dem Koordinatentupel im ECEF-System, $u=u^{\mathrm{ECEF}}$.
Im NED($\sigma$)-System seien die Koordinaten mit $u^{\mathrm{NED}(\sigma)}$ bezeichnet, und in den geographischen Polarkoordinaten entsprechend mit $u^{\mathrm{geo}(\sigma)}$.
Für die Basiswechsel folgt aus Gl.~(\ref{eq:D_Phi})
\begin{align}
	u^{\mathrm{NED}(\sigma)} &= 
	\mathfrak{D}^{\mathrm{NED}(\sigma)}_{\mathrm{geo}(\sigma)}
	\cdot
	u^{\mathrm{geo}(\sigma)}
\end{align}
mit 
\begin{align}
	\mathfrak{D}^{\mathrm{NED}(\sigma)}_{\mathrm{geo}(\sigma)} &= 
	\begin{pmatrix}
		M(\phi) - h & 0 & 0 \\
		0 & [N(\phi) - h]\cos\phi & 0 \\
		0 & 0 & 1				
	\end{pmatrix}
\end{align}
und
\begin{align}
	u^{\mathrm{ECEF}} &= 
	\mathfrak{C}^{\mathrm{ECEF}}_{\mathrm{NED}(\sigma)}
	\cdot
	u^{\mathrm{NED}(\sigma)}
\end{align}
mit 
\begin{align}
	\mathfrak{C}^{\mathrm{ECEF}}_{\mathrm{NED}(\sigma)} &= 
	\begin{pmatrix}
		-\sin\phi\cos\lambda & -\sin\lambda & -\cos\phi\cos\lambda\\
		-\sin\phi\sin\lambda & \cos\lambda & -\cos\phi\sin\lambda\\
		\cos\phi & 0 & -\sin\phi
	\end{pmatrix}
	\in\mathrm{SO}(3)
	\ .
\end{align}
Durch Komposition erhalten wir also
\begin{align}
\label{eq:D_Phi_decomposition}
	D\Phi(\sigma; \nul)
	&=
	\mathfrak{C}^{\mathrm{ECEF}}_{\mathrm{NED}(\sigma)}
	\cdot
	\mathfrak{D}^{\mathrm{NED}(\sigma)}_{\mathrm{geo}(\sigma)}
	\ .
\end{align}
Die Angabe eines Vektors in Koordinaten bezüglich der nicht normierten Basis $(\partial/\partial\phi,\partial/\partial\lambda,\partial/\partial h)$ ist in der Praxis selten interessant.


\section{Transportrate}

\subsection{Definition}

Sei
\begin{align}
	t \mapsto r(t)
\end{align}
mit $t$ aus einem offenen, Null umfassenden Intervall ein Keim einer glatten Trajektorie in unserem Anschauungsraum.
Zum Zeitpunkt~$t=0$ sei $r(0)=r_0$ und $\dot{r}(0)=v_0$.
Wir betrachten den Fall, in dem $r_0$ durch geographische Koordinaten $\sigma_0=(\phi_0, \lambda_0, h_0)=\Phi^{-1}(r_0)$ vorliegt und $v_0$ in NED-Koordinaten zum Punkt $\sigma_0$. 
Das Koeffiziententupel nennen wir entsprechend $v_0^{\mathrm{NED}(\sigma_0)}$

Nach einer (kurzen) Zeit~$\Delta t$ hat die Trajektorie den Punkt $r(\Delta t) = r_0 + v_0\Delta t + \mathrm{o}(\Delta t)$ erreicht.
Das NED-Koordinatensystem zum Punkt $r(\Delta t)$ ist in Bezug auf jenes zum Punkt $r_0$ im Allgemeinen verdreht.
Die NED-Koordinaten eines Vektorfeldes am dann aktuellen Punkts $r(\Delta t)$ weichen also selbst im Falle eines zeitunabhängigen homogenen Vektorfeldes von jenen im Punkte $r_0$ ab.
Dies trifft insbesondere auch auf die Geschwindigkeit zu.

Ziel ist es, die \mydef{Transportrate}~$\omega_{\mathrm{tr}}$ zu ermitteln, mit welcher NED-Koordinaten längs einer Trajektorie rotieren.
Für eine in geographischen Koordinaten gegebene Trajektorie $t \mapsto \sigma(t)$ definieren wir sie als die schiefsymmetrische Matrix
\begin{align}
\label{eq:omega_tr_def}
	\Omega_{\mathrm{tr}}
	&=
	\left[
	\mathfrak{C}^{\mathrm{ECEF}}_{\mathrm{NED}(\sigma)}
	\right]\trans
	\cdot
	\frac{d}{dt}
	\mathfrak{C}^{\mathrm{ECEF}}_{\mathrm{NED}(\sigma)}
	\in\mathfrak{so}(3)
\end{align}
oder als (axialen) Vektor durch die Identifikation~$\Omega_{\mathrm{tr}}=\omega_{\mathrm{tr}}\times\nul$.

Mit Hilfe der Transportrate lässt sich die Zeitentwicklung der in laufenden NED-Koordinaten zum jeweils aktuellen Punkt~$\sigma=\sigma(t)$ angegebenen Geschwindigkeit~$v^{\mathrm{NED}(\sigma)}=\mathfrak{C}_{\mathrm{ECEF}}^{\mathrm{NED}(\sigma)}\cdot \dot{r}$ bequem schreiben.
Mit
\begin{align}
	\frac{d}{dt}
	v^{\mathrm{NED}(\sigma)}
	&=
	\frac{d}{dt}
	\left(
	\left[
		\mathfrak{C}^{\mathrm{ECEF}}_{\mathrm{NED}(\sigma)}
	\right]\trans
	\cdot
	\dot{r}
	\right)
	\nonumber\\
	&=
	\left(
	\frac{d}{dt}
	\left[
		\mathfrak{C}^{\mathrm{ECEF}}_{\mathrm{NED}(\sigma)}
	\right]\trans
	\right)
	\cdot
	\dot{r}
	+
	\left[
		\mathfrak{C}^{\mathrm{ECEF}}_{\mathrm{NED}(\sigma)}
	\right]\trans
	\cdot
	\ddot{r}		
	\ .		
\end{align}
folgt
\begin{align}
	\frac{d}{dt}
	v^{\mathrm{NED}(\sigma)}
	&=
	-\omega_{\mathrm{tr}}
	\times
	v^{\mathrm{NED}(\sigma)}
	+ \left[
		\ddot{r}
	\right]^{\mathrm{NED}(\sigma)}
	\ .	
\end{align}
Der zweite Term ist darin die im NED-System ausgedrückte Beschleunigung der Trajektorie.

\subsection{Berechnung der Transportrate}

Die durch Gl.~(\ref{eq:omega_tr_def}) definierte Transportrate~$\omega_{\mathrm{tr}}$ lässt sich unter Zuhilfenahme der in den vorigen Abschnitten erhaltenen Formeln leicht berechnen. 
Ohne Einschränkung betrachten wir dabei den Zeitpunkt~$t=0$.

Die Zeitableitung in Gl.~(\ref{eq:omega_tr_def}) führen wir zunächst mit Hilfe der Kettenregel aus,
\begin{align}
\label{eq:dC_dt}
	\left.
	\frac{d}{dt}
	\mathfrak{C}^{\mathrm{ECEF}}_{\mathrm{NED}(\sigma)}
	\right|_{t=0}
	&=
	D\mathfrak{C}^{\mathrm{ECEF}}_{\mathrm{NED}}
	\left(
		\sigma_0;
	\left.
	\frac{d\sigma}{dt}
	\right|_{t=0}		
	\right)
	\nonumber\\
	&=
	D\mathfrak{C}^{\mathrm{ECEF}}_{\mathrm{NED}}
	\left(
		\sigma_0;
		D\Phi^{-1}\left(r_0; 	
		\left.
			\frac{dr}{dt}
		\right|_{t=0}	
		\right)		
	\right)	
	\nonumber\\
	&=
	D\mathfrak{C}^{\mathrm{ECEF}}_{\mathrm{NED}}
	\left(
		\sigma_0;
		D\Phi^{-1}\left(r_0; 	
		\mathfrak{C}^{\mathrm{ECEF}}_{\mathrm{NED}(\sigma_0)}
		\cdot
		v^{\mathrm{NED}(\sigma_0)}
		\right)		
	\right)
	\ .	
\end{align}
Die Ableitung $D\mathfrak{C}^{\mathrm{ECEF}}_{\mathrm{NED}}$ können wir nun durch
\begin{align}
\label{eq:D_C}
	&
	\left[
	\mathfrak{C}^{\mathrm{ECEF}}_{\mathrm{NED}(\sigma_0)}
	\right]\trans
	\cdot
	D\mathfrak{C}^{\mathrm{ECEF}}_{\mathrm{NED}}
	\left(
		\sigma_0;
		\nul
	\right)
	\nonumber\\
	&=
	\left[
		\mathfrak{C}^{\mathrm{ECEF}}_{\mathrm{NED}(\sigma_0)}
	\right]\trans
	\left.
	\frac{\partial \mathfrak{C}^{\mathrm{ECEF}}_{\mathrm{NED}(\sigma)}}
	{\partial\phi}
	\right|_{\sigma_0}
	\ d\phi
	+
	\left[
		\mathfrak{C}^{\mathrm{ECEF}}_{\mathrm{NED}(\sigma_0)}
	\right]\trans	
	\cdot
	\left.
	\frac{\partial \mathfrak{C}^{\mathrm{ECEF}}_{\mathrm{NED}(\sigma)}}
	{\partial\lambda}
	\right|_{\sigma_0}
	\ d\lambda
	\nonumber\\
	&=
	\begin{pmatrix}
		0 & 0 & -1\\
		0 & 0 & 0 \\
		1 & 0 & 0
	\end{pmatrix}		
	\ d\phi
	+
	\begin{pmatrix}
		0 & \sin\phi_0 & 0 \\
		-\sin\phi_0 & 0 & -\cos\phi_0 \\
		0 & \cos\phi_0 & 0
	\end{pmatrix}		
	\ d\lambda
\end{align}
näher bestimmen.
Für das Richtungsargument im letzten Schritt der Gl.~(\ref{eq:dC_dt}) verwenden wir den Umkehrsatz und Gl.~(\ref{eq:D_Phi_decomposition}),
\begin{align}
	D\Phi^{-1}(r_0; \nul)
	&=
	\left[
		D\Phi(\sigma_0; \nul)
	\right]^{-1} 
	=
	\left[
		\mathfrak{D}^{\mathrm{NED}(\sigma_0)}_{\mathrm{geo}(\sigma_0)}
	\right]^{-1}
	\cdot
	\left[
		\mathfrak{C}^{\mathrm{ECEF}}_{\mathrm{NED}(\sigma_0)}
	\right]^{-1}
	\ ,
\end{align}
und erhalten schließlich
\begin{align}
\label{eq:D_Phi_v}
	&D\Phi^{-1}\left(r_0; 	
	\mathfrak{C}^{\mathrm{ECEF}}_{\mathrm{NED}(\sigma_0)}
	\cdot
	v^{\mathrm{NED}(\sigma_0)}
	\right)
	=
	\left[
		\mathfrak{D}^{\mathrm{NED}(\sigma_0)}_{\mathrm{geo}(\sigma_0)}
	\right]^{-1}
	\cdot
	v^{\mathrm{NED}(\sigma_0)}
	\nonumber\\
	&=
	\frac{v_{0,\mathrm{N}}^{\mathrm{NED}(\sigma_0)}}
	     {M(\phi_0) - h}
	\frac{\partial}{\partial\phi}
	+
	\frac{v_{0,\mathrm{E}}^{\mathrm{NED}(\sigma_0)}}
	     {[N(\phi_0) - h]\cos\phi_0}
	\frac{\partial}{\partial\lambda}
	+
	v_{0,\mathrm{D}}^{\mathrm{NED}(\sigma_0)}
	\frac{\partial}{\partial h}
	\ .
\end{align}
Dabei bezeichnen die Indizes der  $\mathrm{N}$, $\mathrm{E}$ und $\mathrm{D}$ die Nord-, Ost- und Untenkomponenten.

Einsetzen von Gl.~(\ref{eq:D_Phi_v}) in Gl.~(\ref{eq:D_C}) liefert nun die gesuchte Transportrate
\begin{align}
	\Omega_{\mathrm{tr}}(0)
	&=
	\frac{v_{0,\mathrm{N}}^{\mathrm{NED}(\sigma_0)}}
         {M(\phi_0) - h}
	\begin{pmatrix}
		0 & 0 & -1\\
		0 & 0 & 0 \\
		1 & 0 & 0
	\end{pmatrix}
	+
	\frac{v_{0,\mathrm{E}}^{\mathrm{NED}(\sigma_0)}}
         {N(\phi_0) - h}
	\begin{pmatrix}
		0 & \tan\phi_0 & 0 \\
		-\tan\phi_0 & 0 & -1 \\
		0 & 1 & 0
	\end{pmatrix}
\end{align}
zum Zeitpunkt~$t=0$ als schiefsymmetrische Matrix.

\subsection{Zusammenfassung}

Für eine gegebene geographische Position $\sigma=(\phi,\lambda,h)$ und einer in zu $\sigma$ gehörenden NED-Koordinaten gegebenen Geschwindigkeit
$v^{\mathrm{NED}(\sigma)}$ beträgt die (im selben NED-System ausgedrückte) Transportrate
\begin{align}
	\omega_{\mathrm{tr}}
	&=         
	\frac{v_{\mathrm{E}}^{\mathrm{NED}(\sigma)}}
         {N(\phi) - h}
	\begin{pmatrix}
		1 \\ 0 \\ -\tan\phi
	\end{pmatrix}
	+
	\frac{v_{\mathrm{N}}^{\mathrm{NED}(\sigma)}}
         {M(\phi) - h}
	\begin{pmatrix}
		0 \\ -1 \\ 0
	\end{pmatrix}    	
	\ .
\end{align}
Die Krümmungsradien $N(\phi)$ und $M(\phi)$ sind in Gln.~(\ref{eq:N}) und (\ref{eq:M}) definiert worden.

\end{document}